\documentclass[a4paper,12pt]{article}
\usepackage[utf8]{inputenc}
\usepackage{enumitem}
\usepackage[italian]{babel}
\usepackage[a4paper]{geometry}
\usepackage{amssymb,amsmath,amsthm}
\newcommand{\tto} {\leftrightarrow}

\title{IMPLEMENTAZIONE IN JAVA DEL METODO DI RISOLUZIONE PER LA LOGICA CLASSICA, ED ESTENSIONE A LOGICHE MODALI}
\author{Nicolò Iaccarino - 903870}
\date{Giugno 2024}

\begin{document}

\maketitle

\section{Ente presso cui è stato svolto il lavoro di stage}
Lo stage è stato svolto internamente all'Università degli Studi di Milano, sotto la supervisione del prof. Camillo Fiorentini.

\section{Contesto iniziale}
Il contesto è quello della logica matematica, in particolare la logica classica e le logiche modali non-normali. La situazione da affrontare è quella di automatizzare il calcolo della soddisfacibilità di formule logiche, utile in vari ambiti dell'informatica.

\section{Obiettivi del lavoro}
L'obiettivo del lavoro è quello di Implementare in Java il metodo di risoluzione per la logica classica ed estenderlo a logiche modali non-normali. Inoltre, l'obiettivo è anche quello di implementare le formule e di convertirle in \emph{Forma normale congiuntiva} (CNF) tramite il processo di \textbf{clausificazione}; in questo modo è possibile operare il metodo di risoluzione su di esse.

\section{Descrizione lavoro svolto}
\subsection*{Metodo di risoluzione per la logica classica}
inizialmente, Sono state scritte le classi per rappresentare la CNF, ossia le classi per i letterali, per le clausole e per gli insiemi di clausole. Successivamente, è stato scritto il metodo di risoluzione in una classe apposita. Questo metodo prende in input un insieme di clausole e restituisce \texttt{true}, se è soddisfacibile, \texttt{false} se è insoddisfacibile.

Sono state scritte poi le classi che descrivono le formule generiche (atomiche e composte). Fondamentale è il metodo per effettuare la \emph{clausificazione}, che viene realizzata applicando le regole di inferenza per convertire una formula in un insieme di clausole, seguendo un approccio ricorsivo. Infine è stato utilizzato un parser per effettuare il parsing delle formule.

L'utilizzo pratico del metodo di risoluzione che è stato implementato, è quello di verificare la validità di una formula $F$ letta da \emph{Standard Input}.

\subsection*{Metodo di risoluzione per le logiche modali}
Per quanto riguarda le logiche modali, sono state scritte anche in questo caso le classi per la CNF, tenendo conto che questa volta i letterali possono essere \emph{proposizionali} o \emph{modali}, e le clausole possono essere \emph{locali} o \emph{globali}.

\`E stato implementato il metodo di risoluzione che applica le regole $RES_E$, ossia le regole per la logica modale non-normale minimale \textbf{E} (presentate nell'articolo \cite{Articolo_resolution}); queste regole sono \textbf{LRES, GRES, G2L, LERES, GERES}.

Per quanto riguarda le formule, è stato seguito lo stesso approccio per la logica classica, tenendo conto del connettivo modale \emph{box} ($\Box$). La clausificazione di una formula $\phi$ viene effettuata con le funzioni $\eta$ e \textbf{R}; queste funzioni vengono spiegate in \cite{Articolo_resolution}.
Tutto il resto (parsing, ecc\dots) è stato implementato come nella logica classica.

\section{Tecnologie coinvolte}
La principale tecnologia utilizzata è il linguaggio di programmazione \textbf{Java}, fondamentale per scrivere l'algoritmo che implementa il metodo di risoluzione e anche gli oggetti necessari per il suo funzionamento. Inoltre è stato utilizzato il parser \textbf{ANTLR4} per effettuare il parsing delle formule logiche a partire dalla loro rappresentazione testuale. Infine, è stato utilizzato il framework \textbf{Junit} per poter realizzare il testing del metodo di risoluzione.

\section{Competenze e risultati raggiunti}
Alla fine del lavoro siamo riusciti a verificare la validità di tautologie notevoli, come il principio del terzo escluso, il modus ponens, e le leggi di De Morgan.

Per quanto riguarda le logiche modali, siamo riusciti a dimostrare la validità della regola di congruenza \textbf{RE}: da $\phi \leftrightarrow \psi$ si inferisce $\Box\phi \leftrightarrow \Box\psi$, dove $\phi$ e $\psi$ sono formule.

% \newpage
\bibliographystyle{amsalpha}
\bibliography{riassunto}

\end{document}
